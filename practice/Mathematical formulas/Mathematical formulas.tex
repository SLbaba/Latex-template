\documentclass{article}
\usepackage{ctex}
\usepackage{amsmath}   %equation*环境需要amsmath宏包
\usepackage{amssymb} 
\begin{document}
	\section{简介}
	\LaTeX{}将排版内容分为文本模式和数学模式。文本模式用于普通文本排版, 数学模式用于数学公式排版。
	\section{行内公式}
	\subsection{美元符号}
	交换律是 $a+b=b+a$,如$1+2=2+1=3$
	\subsection{小括号}
	交换律是 \(a+b=b+a\),如\(1+2=2+1=3\)
	\subsection{math环境}
	交换律是 \begin{math}a+b=b+a\end{math}
	,如\begin{math}1+2=2+1=3\end{math}
	\section{上下标}
	\subsection{上标}
	\LaTeX{}中上标用\^{} 
	
	$3x{20}-x+2=0$
	
	$3x^{3x{20}-x+2=0}-x+2=0$
	\subsection{下标}
	
	\LaTeX{}中下表用\_{}
	
	$a_0$,$a_1$,$a_2$
	
	$a_0$,$a_1$,$a_2$,.....,$a_{3x^{20}-x+2}$
	
	\section{希腊字母}
	$\alpha$
	$\beta$
	$\gamma$
	$\epsilon$
	$\pi$
	$\omega$
	
	$\Gamma$
	$\Delta$
	$\Theta$
	$\Pi$  %跟前面那个有区别,这个是大写的,上面那个是小写的
	$\Omega$

	$\alpha^2 + \beta^2 = \gamma^2$
	
	\section{数学字体}
	LaTeX 中的 \textbackslash mathbb 是一个数学字体命令,它用于显示黑体字符。它主要用于表示数学中的整数、实数、复数等数学符号。同时,它也可以用于表示其他字母,并且您可以通过在命令后面添加数字来自定义字符的大小。因此,\textbackslash mathbb 命令是 LaTeX 中一个非常强大且灵活的命令,可以用于创建许多自定义的数学符号。
    
    \subsection{表示常用的集合}
   	\begin{align*}
   		&\text{整数集} \mathbb{Z} \\
   		&\text{自然数集} \mathbb{N} \\
   		&\text{有理数集} \mathbb{Q} \\
   		&\text{实数集} \mathbb{R} \\
   		&\text{复数集} \mathbb{C} 
   	\end{align*}
    \subsection{自定义数学符号}
    \begin{align*}
		& \mathbb{A} \\
		& \mathbb{B} \\
		& \mathbb{C} \\
		& \mathbb{D} 
    \end{align*}

	\section{数学函数}
	$\log$
	$\sin$
	$\cos$
	$\arcsin$
	$\arccos$
	$\ln$
	
	$\sin^2 x + \cos^2 x = 1$
	$y = \arcsin x$
	
	$y = \ln x^{100}$
	
	$\sqrt{2}$
	$\sqrt{x^2 + y^2}$
	$\sqrt{2 + \sqrt{x^2 + 4}}$
	$\sqrt[5]{2 + \sqrt[2]{x^{10}+7}}$
	\section{分式}
	$\frac{x}{x^2+x+1}$
	
	$\frac{\sqrt{x-1}}{\sqrt{x+1}}$
	
	$\frac{1}{1+ \frac{1}{x}}$
	
	$\sqrt{\frac{x}{x^2+x+1}}$
	\section{行间公式}
	
	\subsection{美元符号}
	交换律是
	$$a +b=b+a$$
	如
	$$1+2=2+1=3$$
	\subsection{中括号}
	交换律是
	\[a+b=b+a\]
	如
	\[1+2=2+1=3\]
	\subsection{displaymath环境}
	交换律是
	\begin{displaymath}	a+b=b+a \end{displaymath}
	如
	\begin{displaymath}1+2=2+1=3\end{displaymath}
	\subsection{自动编号公式equation环境}
	交换律见式\ref{eq:commutative1}
	\begin{equation}
		a+b=b+a   \label{eq:commutative1}
	\end{equation}
	\subsection{不编号公式equation*环境}
	交换律见式\ref{eq:commutative2}
	\begin{equation*}
		a+b=b+a \label{eq:commutative2}
	\end{equation*}

	交换律见式\ref{eq:commutative3}
	\begin{equation}
		a+b=b+a \label{eq:commutative3}
	\end{equation}
	公式的编号与交叉引用也是自动实现的,大家在排版中,要习惯于采用自动化的方式处理诸如图,表,公式的编号与交叉引用。再如公式\ref{eq:other}
	\begin{equation}
		x^5-7x^3+4x \label{eq:other}
	\end{equation}
	
	\section{多行数学公式}
	\subsection{gather环境}
	gather(带编号)和gather*(不带编号)环境(可以用\textbackslash \textbackslash 换行)
	
	gather 环境是用来将多行公式集中显示的,它们都将被居中显示在同一行上。
	\subsubsection{自动编号公式gather环境}
	使\textbackslash notage命令可以阻止编号
	\begin{gather}
	 	a + b = b + a \\
	 	ba= ab \\
	 	3 \times 5 = 5 \times 3 \notag 
	\end{gather}
	\subsubsection{不编号公式gather*环境}
	\begin{gather*}
		3+5 = 5+3 = 8 \\
		3 \times 5 = 5 \times 3 
	\end{gather*}
	
	\subsection{align环境}
	align 环境是用来对齐多行公式的,每一行都可以单独对齐,可以使用 \& 符号来设置对齐位置。
	
	
	\subsection{自动编号公式的align环境}
	align环境下的公式是带编号的
	\begin{align}
		% 等号左边对齐
		x_{1000000001} &= \cos y+10 \\
		y &= \ln x^{y^10+y^2+100}+20 \\
		% 等号右边对齐
		\cos \sin \tan (x_{0}^{20} + 100) =& \ln \sin \cos z^10 \\
		y =& x + b \\
		% 左对齐
		&\cos \sin \tan (x_{0}^{20} + 100) = \ln \sin \cos z^10 \\
		&y = x + b
	\end{align}
	
	
	\subsection{不编号公式align*环境}
	align*环境下的公式是不带编号的
	\begin{align*}
		y = x^3 + x^2 + d& \\
		y = x^2 + c& 
	\end{align*}
	\subsection{align环境与gather环境的区别}
	如果你需要对齐多行公式,可以使用 align 环境;如果你需要将多行公式集中在一起显示,则可以使用 gather 环境。
	
	\subsection{split环境}
	LaTeX 中的 split 环境是一种分段公式的环境,它可以将一个公式拆分成多个部分,并在不同的行中显示。这样,每一部分的公式都在独立的一行中显示,便于阅读。
	
	split环境 (对齐采用align环境的方式,编号在中间)
	
	\begin{equation}
		\begin{split}
			\cos 2x &= \cos ^ 2 x - \sin^ 2 x \\
			&= 2\cos^2 x - 1
		\end{split}
	\end{equation}

	\subsection{cases环境}
	\begin{equation*} 
		\begin{cases}
			a_1 & b_1 \\
			a_2 & b_2 \\
			... & ... \\
			a_n & b_n
		\end{cases}
	\end{equation*}

	\begin{equation}
		D(x) = \\
		\begin{cases}
			1, & \text{如果} x \in \mathbb{Q} \\
			0, & \text{如果} x \in \mathbb{R} \setminus \mathbb{Q}
		\end{cases}
	\end{equation}
\end{document}