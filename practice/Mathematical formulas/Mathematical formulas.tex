\documentclass{article}
\usepackage{ctex}
\usepackage{amsmath}   %equation*环境需要amsmath宏包
\begin{document}
	\section{简介}
	\LaTeX{}将排版内容分为文本模式和数学模式。文本模式用于普通文本排版, 数学模式用于数学公式排版。
	\section{行内公式}
	\subsection{美元符号}
	交换律是 $a+b=b+a$,如$1+2=2+1=3$
	\subsection{小括号}
	交换律是 \(a+b=b+a\),如\(1+2=2+1=3\)
	\subsection{math环境}
	交换律是 \begin{math}a+b=b+a\end{math}
	,如\begin{math}1+2=2+1=3\end{math}
	\section{上下标}
	\subsection{上标}
	\LaTeX{}中上标用\^{} 
	
	$3x{20}-x+2=0$
	
	$3x^{3x{20}-x+2=0}-x+2=0$
	\subsection{下标}
	
	\LaTeX{}中下表用\_{}
	
	$a_0$,$a_1$,$a_2$
	
	$a_0$,$a_1$,$a_2$,.....,$a_{3x^{20}-x+2}$
	
	\section{希腊字母}
	$\alpha$
	$\beta$
	$\gamma$
	$\epsilon$
	$\pi$
	$\omega$
	
	$\Gamma$
	$\Delta$
	$\Theta$
	$\Pi$  %跟前面那个有区别,这个是大写的,上面那个是小写的
	$\Omega$
	
	$\alpha^2 + \beta^2 = \gamma^2$
	\section{数学函数}
	$\log$
	$\sin$
	$\cos$
	$\arcsin$
	$\arccos$
	$\ln$
	
	$\sin^2 x + \cos^2 x = 1$
	$y = \arcsin x$
	
	$y = \ln x^{100}$
	
	$\sqrt{2}$
	$\sqrt{x^2 + y^2}$
	$\sqrt{2 + \sqrt{x^2 + 4}}$
	$\sqrt[5]{2 + \sqrt[2]{x^{10}+7}}$
	\section{分式}
	$\frac{x}{x^2+x+1}$
	
	$\frac{\sqrt{x-1}}{\sqrt{x+1}}$
	
	$\frac{1}{1+ \frac{1}{x}}$
	
	$\sqrt{\frac{x}{x^2+x+1}}$
	\section{行间公式}
	
	\subsection{美元符号}
	交换律是
	$$a +b=b+a$$
	如
	$$1+2=2+1=3$$
	\subsection{中括号}
	交换律是
	\[a+b=b+a\]
	如
	\[1+2=2+1=3\]
	\subsection{displaymath环境}
	交换律是
	\begin{displaymath}	a+b=b+a \end{displaymath}
	如
	\begin{displaymath}1+2=2+1=3\end{displaymath}
	\subsection{自动编号公式equation环境}
	交换律见式\ref{eq:commutative1}
	\begin{equation}
		a+b=b+a   \label{eq:commutative1}
	\end{equation}
	\subsection{不编号公式equation*环境}
	交换律见式\ref{eq:commutative2}
	\begin{equation*}
		a+b=b+a \label{eq:commutative2}
	\end{equation*}

	交换律见式\ref{eq:commutative3}
	\begin{equation}
		a+b=b+a \label{eq:commutative3}
	\end{equation}
	公式的编号与交叉引用也是自动实现的,大家在排版中,要习惯于采用自动化的方式处理诸如图,表,公式的编号与交叉引用。再如公式\ref{eq:other}
	\begin{equation}
		x^5-7x^3+4x \label{eq:other}
	\end{equation}
\end{document}