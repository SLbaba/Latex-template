\documentclass{article}
\usepackage{ctex}

\usepackage{graphicx} %引入插入图片的宏包
\graphicspath{{figure/}} %图片在当前目录下的figure目录
%语法 \includegraphics[大小,旋转角度等]{文件名}
\begin{document}
	
	\section{\textbackslash includegraphics命令}
	\subsection{\textbackslash textwidth命令}
	\textbackslash textwidth 是 LaTeX 中的一个变量,表示文档中文本宽度的尺寸。它是在文档类型定义中确定的,并可以在文档的其他部分中使用。例如,在文本框内的文本的宽度可以通过将其与 \textbackslash textwidth 进行比较来确定。
	
	\includegraphics[width=0.125\textwidth]{test.jpg}
	
	\includegraphics[width=0.25\textwidth]{test.jpg}
	
	\includegraphics[width=0.5\textwidth]{test.jpg}
	
	\includegraphics[width=1\textwidth]{test.jpg}
	
	\subsection{\textbackslash textheight命令}
	\textbackslash textheight 是 LaTeX 中的一个变量,表示文档中文本高度的尺寸。它是在文档类型定义中确定的,并可以在文档的其他部分中使用。例如,在文本框内的文本的高度可以通过将其与 \textbackslash textheight 进行比较来确定。
	
	\includegraphics[height=0.125\textheight]{test.jpg}
	
	\includegraphics[height=0.25\textheight]{test.jpg}
	
	\includegraphics[height=0.5\textheight]{test.jpg}
	
	\includegraphics[height=1\textheight]{test.jpg}
	
	\subsection{旋转角度angle参数}
	\includegraphics[angle=90]{test.jpg}
	
	\includegraphics[angle=180]{test.jpg}
	
	\includegraphics[angle=-90]{test.jpg}
	
	\includegraphics[angle=-180]{test.jpg}
	
	\subsection{指定height和width}
	注意单位是cm.
	\subsubsection{指定height}
	
	\includegraphics[height=1cm]{test.jpg}
	
	\includegraphics[height=2cm]{test.jpg}
	
	\includegraphics[height=3cm]{test.jpg}
	
	\includegraphics[height=4cm]{test.jpg}
	
	\includegraphics[height=10cm]{test.jpg}
	\subsubsection{指定width}
	
	\includegraphics[width=1cm]{test.jpg}
	
	\includegraphics[width=2cm]{test.jpg}
	
	\includegraphics[width=3cm]{test.jpg}
	
	\includegraphics[width=4cm]{test.jpg}
	
	\includegraphics[width=5cm]{test.jpg}	
	
	\includegraphics[width=10cm]{test.jpg}
	
\end{document}